%!TEX program = xelatex
\documentclass[letterpaper,12pt]{exam}
\usepackage{../videoNotes}
\usepackage{xcolor}
\usepackage[dvipsnames]{xcolor}
\usepackage{soul}

\usepackage{draftwatermark}
\SetWatermarkText{Draft}
\SetWatermarkScale{1.5}
\SetWatermarkColor{red!20}


\newcommand{\unit}{Unit 07}
\pagestyle{headandfoot}
\firstpageheader{CSC 264 \semester\ \  \unit}{}{Name: $\rule{6cm}{0.15mm}$}
\runningheader{CSC 264 \semester}{\unit}{Page \thepage\ of \numpages}
\firstpagefooter{}{}{}
\runningfooter{}{}{}

\begin{document}

%\underconstruction

\par{\fontfamily{qzc}\selectfont\textbf{}}
\begin{questions}

\section*{\unit\_010 -- C Functions }
\begin{samepage}
    \question Why is C a "low level, high level language?"
    \vspace{5mm}
\end{samepage}
\par
\begin{samepage}
    \question What does the following code mean in C?  How many bytes would be needed (remember to include the null character)
\begin{verbatim}
    char * s = "Hello"
\end{verbatim}
    \vspace{5mm}
\end{samepage}
\par
 \begin{samepage}
     \question  Write the assembler code in the .data section that would create the equivalent of \texttt{\textbf{char * s = "Hello"}}. It was not covered explicitly in the video, but you should be able to figure out the answer.
     \vspace{15mm}
 \end{samepage}
 \par
\begin{samepage}
    \question What does the puts command do?
    \vspace{5mm}
\end{samepage}
\par
 \begin{samepage}
     \question What is \%d in the printf function?
     \vspace{5mm}
 \end{samepage}
 \par
  \begin{samepage}
      \question What is \%x in the printf statement?
      \vspace{5mm}
  \end{samepage}
  \par
   \begin{samepage}
       \question How can the \%x format specifier be used so that it is clear the output is in hex?
       \vspace{5mm}
   \end{samepage}
   \par
    
\rule{0.5\textwidth}{.4pt} %End of section
%----------------------------------
\section*{\unit\_020 -- Farewell, start }

\begin{samepage}
    \question What command may be used for linking to replace the \texttt{\textbf{ld}} command?
    \vspace{5mm}
\end{samepage}
\par
\begin{samepage}
    \question What label needs to be changed in order to use \texttt{\textbf{gcc}} as a linker?
    \vspace{5mm}
\end{samepage}
\par
 
\begin{samepage}
    \question What else can gcc with assembler in addition to linking?
    \vspace{5mm}
\end{samepage}
\par
\begin{samepage}
    \question What is the command you would use to assemble and link your program?
    \vspace{5mm}
\end{samepage}
\par
 \begin{samepage}
     \question What is gasm?   
     \vspace{5mm}
 \end{samepage}
 \par
  \begin{samepage}
      \question In software development, do you need to dig through documentation yourself, or can you depend on Google and AI to do the searching for you?
      \vspace{5mm}
  \end{samepage}
  \par
\rule{0.5\textwidth}{.4pt} %End of section
%----------------------------------
\section*{\unit\_030 -- puts }
\par{\fontfamily{qzc}\selectfont\textbf{Video Length 8:00}}
\begin{samepage}
    \question What does the \texttt{\textbf{puts}} function do in C?  How many arguments does \texttt{\textbf{puts}} take?
    \vspace{5mm}
\end{samepage}
\par
\begin{samepage}
    \question How are parameters passed to functions in x86-64 assembler?
    \vspace{5mm}
\end{samepage}
\par
 
\begin{samepage}
    \question What is the order of register use for sending arguments to function calls?
    \vspace{5mm}
\end{samepage}
\par
\begin{samepage}
    \question Assume the a variable called \texttt{\textbf{sentence}} is defined as a null terminated string.  Write the code needed to call puts from assembler.  Do not worry about the return value.
    \vspace{35mm}
\end{samepage}
\par
 \begin{samepage}
     \question What does the \$ mean in \texttt{\textbf{movq $message, rdi}}?
     \vspace{5mm}
 \end{samepage}
 \par
  

\rule{0.5\textwidth}{.4pt} %End of section
%----------------------------------

%----------------------------------





\end{questions} 
%footer
\begin{center}
    \rule{0.667\textwidth}{.8pt} %End of section
\end{center}


If you have any lingering questions or problems, please write them here or see me.
\vfill
\begin{center}
\includegraphics{../csc264Logo}
\end{center}
\end{document} 