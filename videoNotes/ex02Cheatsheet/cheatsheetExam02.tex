%!TEX program = xelatex
\documentclass[letterpaper,12pt]{exam}
\usepackage{../videoNotes}
\usepackage{float}
\usepackage{xcolor}
\usepackage[dvipsnames]{xcolor}
\usepackage{soul}
%\usepackage{draftwatermark}
\usepackage{multicol}
\usepackage[margin=5mm]{geometry}
\usepackage{fontspec}
\usepackage{tabulary}
\setmainfont{Noto Sans}
\setmonofont{Noto Sans Mono}
%\SetWatermarkText{DRAFT}
%\SetWatermarkScale{1.5}
%\SetWatermarkColor{red!20}
\newcommand{\unit}{Exam 2 Cheatsheet \textcolor{red}{DRAFT}}
\pagestyle{headandfoot}
\begin{document}
\section*{\unit}
\section*{Syscall}
\begin{tabular}{| c | c | c | c| c |}
 \hline
    rax & System Call & rdi & rsi & rdx \\
    \hline
    $0$ & read & file descriptor & buffer & number of bytes \\
    \hline
    $1$ & write & file descriptor & buffer & number of bytes \\
    \hline
    $60$ & exit & exit code & --- & --- \\
    \hline
\end{tabular}
\par
\section*{Calling C library functions}

\begin{itemize}
    \item Parameters are stored in registers in the following order: rdi, rsi, rdx, rcx, r8, r9. (If there are more parameters, they are pushed onto the stack)
    \item Most C functions return an integer or a pointer (which is just an integer).  The return value is placed in the rax register
    \item The called functions may use or destroy the content of the following registers: rax, rcx, rdx, rsi, rdi, r8, r9, r10, r11 
    \item Other registers may be used, but the called function is responsible for saving them.
\vspace{5 mm}

\end{itemize}
\par

\begin{center}
\begin{tabulary}{\textwidth}{C C C C C C C}
\multicolumn{7}{c}{{\huge General Purpose Registers}}\\

64-bit & 32-bit & 16-bit & 8-bit low & 8-bit high & Calling Convention & May be destroyed by called~function?\\
\hline
rax & eax & ax & al & ah & Return Val/Accum & Yes \\
rbx & ebx & bx & bl & bh & --- & No \\
rcx & ecx & cx & cl & ch & 4th argument & Yes \\
rdx & edx & dx & dl & dh & 3rd argument & Yes \\
\hline
rsi & esi & si & sil & -- & 2nd argument & Yes \\
rdi & edi & di & dil & -- & 1st argument & Yes \\
\hline
r8 & r8d & r8w & r8b & -- & 5th argument & Yes \\
r9 & r9d & r9w & r9b & -- & --- & Yes \\
r10 & r10d & r10w & r10b & -- & --- & Yes \\
r11 & r11d & r11w & r11b & -- & --- & Yes \\
r12 & r12d & r12w & r12b & -- & --- & No \\
r13 & r13d & r13w & r14b & -- & --- & No \\
r14 & r14d & r14w & r14b & -- & --- & No \\
r15 & r15d & r15w & r15b & -- & --- & Yes \\
\end{tabulary}
\par 
\vspace{10 mm}
\begin{tabulary}{\textwidth}{L C C C C C}
\multicolumn{6}{c}{{\huge Special Purpose Registers}}\\
Register & 64-bit & 32-bit & 16-bit & 8-bit low & May be destroyed by called function?\\
\hline
\textbf{Stack Pointer} & rsp & esp & sp & spl & No \\
\textbf{Base Pointer} & rbp & ebp & bp & bpl & No \\
\textbf{Instruction Pointer} & rip & eip & ip & -- & \  \\
\textbf{Flags and Conditions} & rflags & eflags & flags & -- & Yes\\
\end{tabulary}
\end{center}



\end{document} 