
%!TEX program = xelatex
\documentclass[letterpaper,12pt]{exam}
\usepackage{../videoNotes}
\usepackage{xcolor}
\usepackage[dvipsnames]{xcolor}
\usepackage{soul}

\newcommand{\unit}{Unit 00}
\pagestyle{headandfoot}
\firstpageheader{CSC 264 \semester\ \  \unit}{}{Name: $\rule{6cm}{0.15mm}$}
\runningheader{CSC 264 \semester}{\unit}{Page \thepage\ of \numpages}
\firstpagefooter{}{}{}
\runningfooter{}{}{}



\begin{document}
\section*{\unit\_000 -- First Assembly Program} 
\par{\fontfamily{qzc}\selectfont\textbf{Video Length 33:00}}
\begin{questions}

\begin{samepage}
    \question What symbols are used to mark a multi-line comment in the `as` assembler?
    \vspace{5mm}
\end{samepage}

\begin{samepage}
    \question What is the name of the section of the program that holds items that look like variables?  (Hint: it starts with a period.)
    \vspace{5mm}
\end{samepage}

\begin{samepage}
    \question What is the name of the section of the program that holds the executable code? (Hint: it also starts with a period.)
    \vspace{5mm}
\end{samepage}

\begin{samepage}
    \question What symbol is used to indicate that the remainder of the line is a comment?
    \vspace{5mm}

\begin{samepage}
    \question The last three lines of the program do not have comment.  Write reasonable comments for each of the lines.
    \begin{Large}
    \begin{verbatim}
movq $60, %rax

movq sum, %rdi
      
syscall
    \end{verbatim}
    \end{Large}
    \vspace{5mm}
\end{samepage}
\end{samepage}
\hrule %End of section
%----------------------------------

\section*{\unit\_010 -- About Assembly}
\par{\fontfamily{qzc}\selectfont\textbf{Video Length 8:27}}
\begin{samepage}
    \question  Is the main purpose of this course to teach you how to write programs in Assembly Language?  Explain your answer.
    \vspace{25mm}
\end{samepage}

\begin{samepage}
    \question 
    \vspace{5mm}
\end{samepage}

\par
\rule{0.5\textwidth}{.4pt} %End of section
%----------------------------------
\section*{\unit\_015 -- A Bit of History}
\par{\fontfamily{qzc}\selectfont\textbf{Video Length 20:30}}
\begin{samepage}
    \question How many bits of data were needed to represent a single decimal digit using the methods used to tablulate the US census inn 1890?
    \vspace{20mm}  Explain why it took that many bits.
\end{samepage}

\begin{samepage}
    \question How many bits are needed to represnt a single decimal diit using Binary Coded Decimal (BCD)
    \vspace{5mm}
\end{samepage}
\begin{samepage}
    \question Why was BCD inefficient?  
    \vspace{5mm}
\end{samepage}
\begin{samepage}
    \question What is the relationship between an electrical relay switch and a bit?
    \vspace{10mm}
\end{samepage}

\begin{samepage}
    \question Relays were replaced by \_\_\_\_\_\_\_\_ tubes which were in turn replaced by \rule{2cm}{0.15mm}.  Later, many transistors and other electrical components were printed on single pieces of silicone to create \rule{2cm}{0.15mm} circuits
    \vspace{5mm}
\end{samepage}
\par
\begin{samepage}
    \question How is the speed of computers related to the physical size of the CPU?
    \vspace{5mm}
\end{samepage}
\par
\begin{samepage}
    \question What is a microprocessor?
    \vspace{5mm}
\end{samepage}
\par

\rule{0.5\textwidth}{.4pt} %End of section
 
\section*{\unit\_020 -- Computer Systems}
\par{\fontfamily{qzc}\selectfont\textbf{Video Length 12:53}}
\begin{samepage}
    \question Who is generally credited with designing the basic architecture of modern computer systems?
    \vspace{5mm}
\end{samepage}
\par

\pagebreak
\begin{samepage}
    \question What are the three main components of the CPU of a CPU?
    
    \vspace{5mm}
    \begin{Large}
         1. \rule{2cm}{0.15mm}  2. \rule{2cm}{0.15mm}  3. \rule{4cm}{0.15mm}
    \end{Large}
\end{samepage}
\par
 \begin{samepage}
     \question What are registers? (You do not need to list specific registers)
     \vspace{5mm}
 \end{samepage}
 \par
 \begin{samepage}
     \question What type of arithmetic is used by the ALU?
     \vspace{5mm}
 \end{samepage}
 \par
   
 \begin{samepage}
     \question What two things are kept in Main Memory or Primary Storage?
     \vspace{5mm}
 \end{samepage}
 \par
 \begin{samepage}
     \question What unpopular opinion makes VonNeuman a controversial figure?
     \vspace{5mm}
 \end{samepage}
 \par
  \begin{samepage}
      \question Does a CPU need to be on a single chip of silicone?
      \vspace{5mm}
  \end{samepage}
  \par
   \begin{samepage}
       \question In old movies and TV shows, CPUs had lots of blinking lights.  What did the blinking lights represent?
       \vspace{5mm}
   \end{samepage}
   \par
   \begin{samepage}
       \question What is a bus?
       \vspace{5mm}
   \end{samepage}
   \par
\section*{\unit\_030 -- Computer architecture}
\par{\fontfamily{qzc}\selectfont\textbf{Video Length 8:13}}
\begin{samepage}
    \question In general terms, what does "Computer Architecture" mean?
    \vspace{5mm}
\end{samepage}
\begin{samepage}
    \question What is the specific computer architecture we will be studying this semester?
    \vspace{5mm}
ar
\end{samepage}
\begin{samepage}
    \question How wide were the registers on 8086 processors
    \vspace{5mm}
\end{samepage}
\par
\begin{samepage}
    \question How wide are the registers on 80386 processors? (Oops, I did not say 80386 in the video, so I will answer this one for you.)
    \par  
    \textcolor{blue}{\LARGE 32 bits}
    
   
\end{samepage}
\par 
\begin{samepage}
    \question How wide are the registers on x86-64 processors? (Yes, it is obvious)
    \vspace{5mm}
\end{samepage}

\begin{samepage}
    \question How are ARM processors different from x86 processors?
    \vspace{5mm}
\end{samepage}
\par
\begin{samepage}
    \question What is "Instruction Set Architecture? Do all CPUs share the same ISA?
    \vspace{15mm}
\end{samepage}
\par

\rule{0.5\textwidth}{.4pt} %End of section

\section*{\unit\_040 -- Machine Language}
\par{\fontfamily{qzc}\selectfont\textbf{Video Length 5:40}}
\begin{samepage}
    \question The assembly language program translates assembly language into \rule{3cm}{0.15mm} language.
    \vspace{5mm}
\end{samepage}
\par

NOTE:  (Not a question).  The program I wrote for the first video is in 64-bit assembler.  When i made this 040 video, I wanted the machine code to be neat in the listing.  I did not like how every line of machine code in 64-bit assembly takes two lines, with the second line being 000000000.  So I sort of cheated and wrote this program using only 16-bit assembler.  In the first program, I used the %rax register.  The "r" indicates it is 64-bit.  In the code I wrote for the 040 video, I used the %ax register, which is the 
lower 16 bits of the %rax register.  The program still works fine on a 64-bit CPU, but it looks nicer when I make a listing.  

\begin{samepage}
    \question What is the relationship between hexadecimal and binary?
    \vspace{5mm}
\end{samepage}
\par
 \begin{samepage}
     \question In the 000 video, I had the following line of code:

     \begin{verbatim}
      movq sum, %rdi       
     \end{verbatim}
     In this video, I wrote the equivalent line of code as:
     \begin{verbatim}
        mov $7, %di
        \end{verbatim}
        Explain the differences between these two lines of code.
     \vspace{5mm}
 \end{samepage}
 \par
  

\rule{0.5\textwidth}{.4pt} %End of section
%----------------------------------
\section*{\unit\_050 -- Computer Languages}
\par{\fontfamily{qzc}\selectfont\textbf{Video Length 5:28}}
\begin{samepage}
    \question How are high-level languages different from low-level languages?
    \vspace{5mm}
\end{samepage}
\par
\begin{samepage}
    \question The PDP-11 had a row of switches.  What did each switch represent?
    \vspace{5mm}
\end{samepage}
\rule{0.5\textwidth}{.4pt} %End of section
%----------------------------------


\section*{\unit\_060 -- Language Translators}
\par{\fontfamily{qzc}\selectfont\textbf{Video Length 8:00}}
\begin{samepage}
    \question I C a compiled language or an interpreted language?
    \vspace{5mm}
\end{samepage}
\par
\begin{samepage}
    \question 
    \vspace{5mm}
\end{samepage}
\par
 \begin{samepage}
     \question 
     \vspace{5mm}
 \end{samepage}
 \begin{samepage}
     \question What is the linux command that prints a program's exit code or error code.
     \vspace{5mm}
 \end{samepage}
 \par
  \begin{samepage}
      \question Is Python a compile language or an interpreted language?
      \vspace{5mm}
  \end{samepage}
  \par
   \begin{samepage}
       \question Is assembler more like a compiled language or an interpreted language?
       \vspace{5mm}
   \end{samepage}
   \par
    
\section*{\unit\_070 -- Assembler}
\par{\fontfamily{qzc}\selectfont\textbf{Video Length 7:58}}
\begin{samepage}
    \question What are the two main dialects of assemblers used for x86 assemblers?  Which are we using this semester? 
    \vspace{5mm}
\end{samepage}
\begin{samepage}
    \question Many assembly language instructions use two operands.  What order do GAS assemblers use?
    \vspace{5mm}
\end{samepage}

\begin{samepage}
     \question We assemble assembly language into machine code.  Can the process be reversed?  When would we want to reverse the process?
     \vspace{5mm}
 \end{samepage}
 

 \rule{0.5\textwidth}{.4pt} %End of section
\section*{\unit\_080 -- Generations of Programming languages}
\par{\fontfamily{qzc}\selectfont\textbf{Video Length 15:08}}
\begin{samepage}
    \question How many bits were in a register 8086 processors?  
    \vspace{5mm}
\end{samepage}
\begin{samepage}
    \question The 8086 could only address 1MB of memory.  Why was it limited to 1MB? (Hint: How many bits were in the Address Bus?)
    \vspace{5mm}
\end{samepage}
\par
\begin{samepage}
    \question What was the 8087 coprocessor used for?
    \vspace{5mm}
\end{samepage}


  \begin{samepage}
      \question How many bits were in the 80386 processor? \rule{1cm}{0.15mm}
        \vspace{5mm}
      \par Was the eax register available on the 80386? \rule{1cm}{0.15mm}
       \vspace{5mm}
      \par Was the rax register available on the 80386? \rule{1cm}{0.15mm}
  \end{samepage}


  %%%%%%%%%%%%%%%%%%%%%%%%%%%%%%%%%%%%%%%%%%%%%%%%%%%%%
\begin{center}
    \rule{0.5\textwidth}{.4pt}
\end{center}
Please write any lingering questions you have here.
  
\rule{0.5\textwidth}{.4pt} %End of section
%----------------------------------
\end{questions}


\end{document}